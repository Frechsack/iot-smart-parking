\chapter{Introduction}
\par
The "Duale Hochschule Baden-Würtemberg" is one of the biggest institution for higher education in Heidenheim an der Brenz located in Germany. The key concept of the "Duale Hochschule" is to integrate academic studies with workplace experience.\footcite[][]{dhbwCooperativeState} 
At the moment more than 2.400 students and 9.000 companies and institutions corporate with each other.\footcite[][]{dhbwCooperativeState} 
Many students and teachers arrive from the surrounding area of Heidenheim. Heidenheim itself is a small city with around 50.000 inhabitants.\footcite[][]{heidenheimZahlenDaten}
To handle the inflow of people and individual traffic by car, Heidenheim offers six parking garages. The Academy itself does not provide parking spaces for students. They have to rely on public parking possibilities.
For teachers the Academy overs a small separated area for multiple parking spaces. Those parking lots are protected by a barriers. To access the parking lots, you have to identify yourself with a valid Academy-ID. Since the Academy can not separate between teachers and students based on their Academy-ID, a lot of students park on those parking lots, even if they are not allowed to do so. In the recent past a lot of teachers complained about the current situation. The problem here is a non existing differentiation between teachers and students.
Since Heidenheim has to deal with a lot of traffic, there is a big rush into the public available parking garages as well. Within those parking garages, the search for an available parking lot, decreases the flow within the parking garage. 
\par
This proposal contains a technical concept how an already existing parking garage can be equipped with an automated access control and a system to navigate incoming cars to the nearest available parking lot within the garage. 
To verify the concept, a prototype of this concept was built with micro-controllers.
